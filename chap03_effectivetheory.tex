
\chapter{Dimension-6 Operators Implementation into WHIZARD}

\more{This chapter describes about the project to implement 6 dimensional operators into WHIZARD.}
The operators are arised from the effective field theory formalism. 
The resourcese I used for this chapter are \mycites{Grzadkowski:2010es}...

\section{Effective Field Theory}
% \section{}
\label{sec:eft}

\more{Some introduction to EFT... }

% EFT, Howard Georgi; 

% Why do we need exactly "Effective Field Theory"? What's special feature of EFT? 
% Little Higgs / EFT both are handeling the hierarchy problem. -> light Higgs
%  the Glashow-Weinberg condition. (GIM mechanism이 계속 작동하도록 하는.. ), 
% 차지가 2/3인 쿼크랑 커플링하는 더블렛이랑, -1/3인 쿼크랑 커플하느 더블렛이랑 구분이 된다면 김 메카니즘이 작동 가능
%  GIM FCNC가 안일어 나게, 룹다이어그램에서 서프레싱 해주는 거래.

Because of renormalization conditions, 
the dimension of the Lagrangian of the Standard Model has 4 space-time dimension.
However, as considering the Standard Model as the low energy effective theory, 
we can build up higher-dimensional operators such as 5- and 6-dimensional operators.

\more{EFT looks like this to me:}
Let's say there is a true happening in the high scale. 
And we observe something in the accessible energy scale but they look different from the true one
because it is converted to combinations of various processes while running down the energy level. 
Although observing different signals, we make a guess about it
by integrating out real vertices and matching and merging the processes. 

\more{For example, tri-linear vector scattering}

\note{In some approach to the effective field theory formalism}, 
the Standard model is regarded as an effective field theory in the low energy level. 
When there are anomalies, EFT can manage them by integrating out them.

\begin{equation} 
\mathcal{L}_{\mathrm{total}} = \mathcal{L}_{\mathrm{SM}} 
+ \sum_{k}\left[ \frac{1}{\Lambda} C_k^{(5)} \mathcal{O}_k^{(5)}
+ \frac{1}{\Lambda^2} C_k^{(6)} \mathcal{O}_k^{(6)} + \mathcal{O}\left(\frac{1}{\Lambda^3}\right) \right]
\end{equation}
where $\cL_{SM}$ is the Lagrangian of the SM and $\Lam$ is the cutoff scale that the SM is recovered from the full theory.
$\cO^{(n)}$ corresponds to $n$-dimensional operator 
and $C_k^{(n)}$ is its dimensionless coupling constant
that will be determined by matching after integrating out the heavy fields. 

Higher-dimensional operators are already derived long time ago by Buchm\"{u}ller and Wyler \mycite{k}.
But in that paper, some operators were considered as duplicated physicswise
because they are removed by the Equations of Motion and do not give contributions to on-shell matrix elements.
Thus, we used GIMR basis in \mycite{Grzadkowski:2010es} and from now on we are following their point of view about the higher dimensional operators. 
% That is  


\paragraph{notation}
\begin{align}
 (D_\mu q)^{\alpha j} = [\delta_{\alpha \beta} \delta_{jk}(\partial_\mu + ig'Y_qB_\mu)
  + ig \delta_{\alpha\beta}S_{jk}^I W_\mu^I + ig_s \delta_{jk} T_{\alpha\beta}^A G_\mu^A] q^{\beta k}\nonumber\\
  \phi^\dagger i \overleftrightarrow{D}_\mu \phi \equiv i\phi^\dagger (D_\mu - \overleftarrow{D}_\mu)\phi~,
~~\phi^\dagger i \overleftrightarrow{D}_\mu^I \phi \equiv 
i\phi^\dagger (\tau^I D_\mu - \overleftarrow{D}_\mu \tau^I)\phi
\end{align}


\section{6-dimensional operators}
In the SM, the renormalisable operators have 4 in spacetime dimension then of course, there is 5-dimensional operator as well.
\paragraph{5-dimensional operator}
Only one exists. Weinberg operator. Majorana term?



\paragraph{Bosonic operators}
And CP-even operators
\begin{enumerate}
 \item $\mathcal{O}_6$ = $(\Phi^\dagger \Phi)^3$
 \item $\mathcal{O}_\Phi$ = $\partial_\mu (\Phi^\dagger \Phi) \partial^\mu (\Phi^\dagger \Phi)$
 \item $\mathcal{O}_T$ = $(\Phi^\dagger D_\mu \Phi) (\Phi^\dagger D^\mu \Phi)$
 \item $\mathcal{O}_{DW}$ = $(\Phi^\dagger \tau^I i \overleftrightarrow{D}^\mu \Phi) (D^\nu W_{\mu\nu})^I$
 \item $\mathcal{O}_{DB}$ = $(\Phi^\dagger \overleftrightarrow{D}^\mu \Phi) (\partial^\nu B_{\mu\nu})$
 \item $\mathcal{O}_{D \Phi W}$ = $i (D^\mu\Phi)^\dagger \tau^I (D^\nu \Phi)W_{\mu\nu}^I$
 \item $\mathcal{O}_{D \Phi B}$ = $i (D^\mu\Phi)^\dagger \tau^I (D^\nu \Phi)B_{\mu\nu}^I$
 \item $\mathcal{O}_{\Phi B}$ = $(\Phi^\dagger \Phi)B_{\mu\nu}B^{\mu\nu}$
 \item $\mathcal{O}_{\Phi G}$ = $(\Phi^\dagger \Phi)G^A_{\mu\nu}G^{A\mu\nu}$
 \item $\mathcal{O}_{G}$ = $f^{ABC} G^{A\nu}_{\mu} G^{B\rho}_{\nu} G^{C\mu}_{\rho}$
 \item $\mathcal{O}_{W}$ = $\epsilon^{IJK} W^{I\nu}_{\mu} W^{J\rho}_{\nu} W^{K\mu}_{\rho}$
\end{enumerate}



\subsection{possible vertices}
Using FeynRules, 
\note{unitarisation scheme?} 
% 이건 정말 생각지도 않아본 파트인데 ... 
\note{yellow report}
% 에는 뭐라고 나와있는지 읽어봐야겠다


\section{WHIZARD}
\subsection*{W Higgs Z Advanced Research D...ragon ;p}
